\documentclass{report} % 10pt
% Packages for extra functionality.
\usepackage{amsmath,amsfonts,amsthm,amssymb}
\usepackage{setspace}
\usepackage{fancyhdr}
\usepackage{lastpage}
\usepackage{extramarks}
\usepackage{chngpage}
\usepackage{soul}
\usepackage[usenames,dvipsnames]{color}
\usepackage{graphicx,float,wrapfig}
\usepackage{subfig}
\usepackage{ifthen}
\usepackage{listings}
\usepackage{courier}
\usepackage{microtype}
\usepackage{appendix}
\usepackage{hyperref}
\usepackage[all]{hypcap}
\usepackage{url}

% Needed for changing the font size in \documentclass{} while
% using the fancyhdr package. Else the warning:
% \headheight is too small (12.0pt):Make it at least 14.49998pt.
\setlength{\headheight}{15pt}

% In case you need to adjust margins:
\topmargin=-0.45in%
\evensidemargin=0in%
\oddsidemargin=0in%
\textwidth=6.5in%
\textheight=9.25in%
\headsep=0.25in%

% Setup the header and footer
\pagestyle{fancy}%
\lhead{\hmwkAuthor}%
\chead{\hmwkClass~\hmwkTitle}%
\rhead{\firstxmark}%
\lfoot{\lastxmark}%
\cfoot{}%
\rfoot{Page\ \thepage\ of\ \protect\pageref{LastPage}}%
\renewcommand\headrulewidth{0.4pt}%
\renewcommand\footrulewidth{0.4pt}%

% Adds a hyperlink to an email address.
\newcommand{\mailto}[2]{\href{mailto:#1}{#2}}

% This is used to trace down (pin point) problems
% in latexing a document:
%\tracingall

%%%%%%%%%%%%%%%%%%%%%%%%%%%%%%%%%%%%%%%%%%%%%%%%%%%%%%%%%%%%%
% Some tools
\newcommand{\enterProblemHeader}[1]{\nobreak\extramarks{#1}{#1 continued on next page\ldots}\nobreak%
                                    \nobreak\extramarks{#1}{#1 continued on next page\ldots}\nobreak}%
\newcommand{\exitProblemHeader}[1]{\nobreak\extramarks{#1}{#1 continued on next page\ldots}\nobreak%
                                   \nobreak\extramarks{#1}{}\nobreak}%

\newlength{\labelLength}
\newcommand{\labelAnswer}[2]
  {\settowidth{\labelLength}{#1}%
   \addtolength{\labelLength}{0.25in}%
   \changetext{}{-\labelLength}{}{}{}%
   \noindent\fbox{\begin{minipage}[c]{\columnwidth}#2\end{minipage}}%
   \marginpar{\fbox{#1}}%

   % We put the blank space above in order to make sure this
   % \marginpar gets correctly placed.
   \changetext{}{+\labelLength}{}{}{}}%

\setcounter{secnumdepth}{0}
\newcommand{\homeworkProblemName}{}%
\newcounter{homeworkProblemCounter}%
\newenvironment{homeworkProblem}[1][Problem \arabic{homeworkProblemCounter}]%
  {\stepcounter{homeworkProblemCounter}%
   \renewcommand{\homeworkProblemName}{#1}%
   \section{\homeworkProblemName}%
   \enterProblemHeader{\homeworkProblemName}}%
  {\exitProblemHeader{\homeworkProblemName}}%

\newcommand{\problemAnswer}[1]
  {\noindent\fbox{\begin{minipage}[c]{\columnwidth}#1\end{minipage}}}%

\newcommand{\problemLAnswer}[1]
  {\labelAnswer{\homeworkProblemName}{#1}}

\newcommand{\homeworkSectionName}{}%
\newlength{\homeworkSectionLabelLength}{}%
\newenvironment{homeworkSection}[1]%
  {% We put this space here to make sure we're not connected to the above.
   % Otherwise the changetext can do funny things to the other margin

   \renewcommand{\homeworkSectionName}{#1}%
   \settowidth{\homeworkSectionLabelLength}{\homeworkSectionName}%
   \addtolength{\homeworkSectionLabelLength}{0.25in}%
   \changetext{}{-\homeworkSectionLabelLength}{}{}{}%
   \subsection{\homeworkSectionName}%
   \enterProblemHeader{\homeworkProblemName\ [\homeworkSectionName]}}%
  {\enterProblemHeader{\homeworkProblemName}%

   % We put the blank space above in order to make sure this margin
   % change doesn't happen too soon (otherwise \sectionAnswer's can
   % get ugly about their \marginpar placement.
   \changetext{}{+\homeworkSectionLabelLength}{}{}{}}%

\newcommand{\sectionAnswer}[1]
  {% We put this space here to make sure we're disconnected from the previous
   % passage

   \noindent\fbox{\begin{minipage}[c]{\columnwidth}#1\end{minipage}}%
   \enterProblemHeader{\homeworkProblemName}\exitProblemHeader{\homeworkProblemName}%
   \marginpar{\fbox{\homeworkSectionName}}%

   % We put the blank space above in order to make sure this
   % \marginpar gets correctly placed.
   }%

% Includes a figure
% The first parameter is the label, which is also the name of the figure
%   with or without the extension (e.g., .eps, .fig, .png, .gif, etc.)
%   IF NO EXTENSION IS GIVEN, LaTeX will look for the most appropriate one.
%   This means that if a DVI (or PS) is being produced, it will look for
%   an eps. If a PDF is being produced, it will look for nearly anything
%   else (gif, jpg, png, et cetera). Because of this, when I generate figures
%   I typically generate an eps and a png to allow me the most flexibility
%   when rendering my document.
% The second parameter is the width of the figure normalized to column width
%   (e.g. 0.5 for half a column, 0.75 for 75% of the column)
% The third parameter is the caption.
\newcommand{\scalefig}[3]{%
  \begin{figure}[ht!]
    % Requires \usepackage{graphicx}
    \centering
    \fbox{%
      \includegraphics[width=#2\columnwidth]{#1}
    }
    %%% I think \captionwidth (see above) can go away as long as
    %%% \centering is above
    %\captionwidth{#2\columnwidth}%
    \caption{#3}
    \label{#1}% chktex 24
  \end{figure}}


% Homework Specific Information
\newcommand{\hmwkTitle}{Homework 1}
\newcommand{\hmwkSubTitle}{}
\newcommand{\hmwkDueDate}{Due Date}
\newcommand{\hmwkClass}{Course Title}
\newcommand{\hmwkClassTime}{}
\newcommand{\hmwkClassInstructor}{Instructor Name}
\newcommand{\hmwkAuthor}{Student Name}
\newcommand{\hmwkAuthorEmail}{Student Email}

% These commands set the document properties for the PDF output. Needs the hyperref package.
\hypersetup
{
    colorlinks,
    linkcolor={black},
    citecolor={black},
    filecolor={black},
    urlcolor={blue},
    pdfauthor={\hmwkAuthor <\mailto{\hmwkAuthorEmail}{\hmwkAuthorEmail}>},
    pdfsubject={\hmwkClass},
    pdftitle={\hmwkTitle},
    pdfkeywords={},
    pdfstartpage={1},
}

%%%%%%%%%%%%%%%%%%%%%%%%%%%%%%%%%%%%%%%%%%%%%%%%%%%%%%%%%%%%%
% Make title
\title{\vspace{2in}\textmd{\textbf{\hmwkClass:\ \hmwkTitle\ifthenelse{\equal{\hmwkSubTitle}{}}{}{\\\hmwkSubTitle}}}\\\normalsize\vspace{0.1in}\small{Due\ on\ \hmwkDueDate}\\\vspace{0.1in}\large{\textit{\hmwkClassInstructor\ \hmwkClassTime}}\vspace{3in}}
\date{}
\author{\hmwkAuthor}
%%%%%%%%%%%%%%%%%%%%%%%%%%%%%%%%%%%%%%%%%%%%%%%%%%%%%%%%%%%%%

\begin{document}
\begin{spacing}{1.1}
\maketitle
% Uncomment the \tableofcontents and \newpage lines to get a Contents page
% Uncomment the \setcounter line as well if you do NOT want subsections
%       listed in Contents
%\setcounter{tocdepth}{1}
%\tableofcontents
%\newpage
%
% When problems are long, it may be desirable to put a \newpage or a
% %\clearpage before each homeworkProblem environment

% This is to add a blank page after the title page. It looks better when
% printing double-sided.
\newpage
\thispagestyle{empty}
\mbox{}
\newpage
\pagenumbering{arabic}

%%%%%%%%%%%%%%%%
% Problem 1.
%%%%%%%%%%%%%%%%
\begin{homeworkProblem}
Consider a discrete-time system given by the controllable and observable $n$th order state space realization
\begin{align}
\label{eq:ss}
\begin{split}
x(k+1) &= Fx(k)+Gu(k) \\
y(k) &= Cx(k) + Du(k)
\end{split}
\end{align}
and consider the following assignments:

\begin{homeworkSection}{1.1}
Give an expression for the impulse response coefficients $g(l)$ in
\begin{align}
\begin{split}
y(k) = G_0(q)u(k) \\
G_0(q) = \sum_{l=0}^\infty g(l)q^{-l}
\end{split}
\end{align}
in terms of the state space matrices $F$, $G$, $C$ and $D$ given in (\ref{eq:ss}).
\newline
\problemAnswer{
To go from a state space to the impulse response representation we apply an impulse as input where
$$u(l) = \begin{cases} 1, & l=0 \\ 0, & l\neq 0 \end{cases}$$
Substituting this into (\ref{eq:ss}) gives
\begin{align*}
y(0) &= Cx(0) + Du(0) = D \\
x(1) &= Fx(0) + Gu(0) = G \\
y(1) &= Cx(1) + Du(1) = CG \\
x(2) &= Fx(1) + Gu(1) = FG \\
y(2) &= Cx(2) + Du(2) = CFG \\
x(3) &= Fx(2) + Gu(2) = F^2G \\
y(3) &= Cx(3) + Du(3) = CF^2G \\
&\vdots \\
g(l) &= \begin{cases} D, & l=0 \\ CF^{l-1}G, & l\neq 0 \end{cases}
\end{align*}
}
\end{homeworkSection}

\begin{homeworkSection}{1.2}
Organize the impulse response coefficients $g(l)$ in a matrix of the form
\begin{align*}
H =
\left[\begin{array}{c c c c}
g(1) & g(2) & \cdots & g(N_2) \\
g(2) & g(3) & \cdots & g(N_2+1) \\
\vdots & \vdots & \ddots & \vdots \\
g(N_1) & g(N_1+1) & \cdots & g(N_1+N_2-1)
\end{array}\right]
\end{align*}
where $N_1,N_2>n$. Knowing that $[F,G,C]$ is controllable/observable, show that $\text{rank}(H)=n$, where $n$ is the order of the state space model.
\newline
\problemAnswer{
Recall that the observability matrix, $\mathcal{O}$, is
$$\mathcal{O} = \left[\begin{array}{c} C \\ CF \\ CF^2 \\ \vdots \\ CF^{N_1-1} \end{array}\right]$$
and the controllability matrix, $\mathcal{C}$, is
$$\mathcal{C} = \left[\begin{array}{c c c c c} B & FB & F^2B & \cdots & F^{N_2-1}B \end{array}\right]$$
Letting $H=\mathcal{O}\cdot\mathcal{C}$ gives
$$H = \left[\begin{array}{c c c c c}
CG & CFG & CF^2G & \cdots & CF^{N_2}G \\
CFG & CF^2G & CF^3G & \cdots & CF^{N_2+1}G \\
CF^2G & CF^3G & CF^4G & \cdots & CF^{N_2+2}G \\
\vdots & \vdots & \vdots & \ddots & \vdots \\
CF^{N_1}G & CF^{N_1+1}G & CF^{N_1+2}G & \cdots & CF^{N_1+N_2-1}G
\end{array}\right]
$$
Then, looking at the Markov parameters we have, using the results from Problem 1.1,
\begin{align*}
g(1) &= CG \\
g(2) &= CFG \\
g(3) &= CF^2G \\
&\vdots \\
g(l) &= CF^{l-1}G
\end{align*}
which are exactly the entries of $H=\mathcal{O}\cdot\mathcal{C}$, which is exactly the same as the matrix $H$ we are being asked to organize. Knowing that $[F,G,C]$ is controllable and observable tells us that $\mathcal{O}$ has full column rank $n$ and $\mathcal{C}$ has full row rank $n$. Since $N_1,N_2>n$ and knowing the rank of $\mathcal{O}$ and $\mathcal{C}$ is only $n$ we can say that $\text{rank}(H)=n$.
}
\end{homeworkSection}
\end{homeworkProblem}

\end{spacing}
% \bibliographystyle{plain}
% \bibliography{mybib}
\end{document}
